%!TEX root = ./thesis-main.tex
\chapter{Implementazione e Verifica}

\section{Componenti grafici}
Lo sviluppo web moderno trae benefici significativi da framework che semplificano la creazione di applicazioni web. Tutte le componenti grafiche all'interno di questo progetto sono state sviluppate attraverso l'utilizzo del framework open-source \textit{KVision}\footnote{\url{https://kvision.gitbook.io/kvision-guide/}}, che permette agli sviluppatori di costruire interfacce web moderne senza utilizzare HTML, CSS o JavaScript. Le interfacce vengono assemblate attraverso la composizione di oggetti pronti all'uso, seguendo un paradigma paragonabile a quello dichiarativo. Il risultato sono gerarchie di componenti che possono essere usate come blocchi costituenti dell'intera interfaccia. In aggiunta, \textit{KVision} presenta supporto integrato per gli store Redux (esplorati in \cref{section:state-management}), per le icone Font Awesome, per Bootstrap e molto altro. Questa libreria sfrutta puramente le capacità del linguaggio Kotlin (compilato per target JS), specialmente con l'utilizzo di \textit{type safe builders}, implementati attraverso \textit{extension functions}. Quello offerto da \textit{KVision} è di fatto un \ac{DSL}, di cui si è fatto ampiamente uso.
Nel \cref{lst:graphic-comp-declaration} è illustrata, a scopo esemplificativo, una versione riadattata della definizione della struttura del contenuto principale della pagina web.

\lstinputlisting[float=htb, language=KVision, label={lst:graphic-comp-declaration}, caption={\ac{DSL} KVision per la struttura del contenuto principale della pagina}]{listings/Content.kt}

Per la maggioranza degli elementi dichiarabili in HTML, esiste una classe corrispondente offerta da \textit{KVision}. Da qui deriva la natura dichiarativa nello sviluppo delle componenti grafiche. Questo avviene grazie all'utilizzo di \textit{extension functions}, funzionalità caratteristica del linguaggio Kotlin che permette di aggiungere nuovi comportamenti a classi esistenti senza dover ereditare da esse o modificarle direttamente. In questo caso, ciò significa che per un elemento del \ac{DOM} come \texttt{div}, il framework fornisce la classe \texttt{Div} alla qual è stata aggiunta l'\textit{extension function} \texttt{div()} che funge da \textit{builder}. L'invocazione di quest'ultima evita la memorizzazione dell'istanza della componente in una variabile separata, e la successiva aggiunta manuale alla lista di figli della classe padre che li contiene.
Nel \cref{lst:graphic-comp-declaration} sono stati creati componenti custom che estendono la classe \texttt{SimplePanel} (classe base per un semplice pannello rappresentabile nella pagina). Si parla di \texttt{SimulationContext}, \texttt{SimulationIndicators} e \texttt{NodeProperties}, come indicato nella sezione \ref{section:page-structure}. In questo caso sono stati aggiunti manualmente come componenti figli delle rispettive classi padre, tramite il metodo \texttt{add()}. Sarebbe stato possibile aggiungere a queste classi una \textit{extension function} per creare un builder. Ogni volta in cui sarebbe stato necessario dichiarare più volte questo tipo di oggetto in diverse parti dell'applicativo, sarebbe bastata una invocazione come nel \cref{lst:graphic-comp-declaration-ext}. In questo caso non era strettamente necessario, perché queste componenti vengono istanziate una volta sola.

\lstinputlisting[float=htb, language=KVision, label={lst:graphic-comp-declaration-ext}, caption={Uso di \textit{type-safe} builders}]{listings/Content_ExtensionFunction.kt}

La manipolazione degli elementi all'interno di un contenitore in una pagina web risulta essere molto onerosa se si fa uso di HTML e CSS puro. La convenienza di utilizzo di questo framework risalta anche nell'esistenza di elementi pronti all'uso come \texttt{hPanel} e \texttt{vPanel}, due costrutti che dispongono, rispettivamente, in orizzontale e in verticale gli elementi aggiunti al loro interno. La disposizione degli oggetti all'interno di questi layout è facilitata perché quest'ultimi rispettano le specifiche della recommendation \textit{CSS Flexible Box Layout} \footnote{\url{https://www.w3.org/TR/css-flexbox/}}.

Inutile dire che la libreria utilizzata offre una vasta gamma di costrutti e metodi che semplificano notevolmente lo sviluppo di interfacce utente, contribuendo così a migliorare l'efficienza complessiva del processo di sviluppo.

\section{Integrazioni di operazioni GraphQL}

\section{Gestione dello stato}\label{section:state-management}
Il framework \textit{KVision}, oltre a fornire strumenti e metodi di programmazione molto robusti e versatili, dona la possibilità di usare tutta la capacità della libreria di \textit{Redux}\footnote{\url{https://redux.js.org/tutorials/essentials/part-1-overview-concepts}}, una libreria open-source per la gestione dello stato delle applicazioni JavaScript. Il fulcro di questa libreria consiste nel cosiddetto \textit{store}, un archivio centralizzato per uno stato che deve essere condiviso in tutta l'applicazione, attraverso l'utilizzo di regole che garantiscono che lo stato possa essere aggiornato solamente in modo prevedibile. Analizziamo ora il funzionamento della gestione dello stato introducendo tutti i concetti chiave: 
\begin{itemize}	
	\item \textbf{State}: ``The source of truth that drives our app'', in altre parole, dati o insieme di dati che influenzano il comportamento o l'aspetto dell'applicazione. 	
	\item \textbf{View}: una descrizione dichiarativa dell'interfaccia utente basata sullo stato attuale.
	\item \textbf{Actions}: usati per descrivere possibili cambiamenti dello stato. Sono oggetti, dotati di un campo che ne indica il tipo, incaricati a indicare l'azione che deve essere eseguita sullo store per cambiarne lo stato. Si può pensare a questo tipo di oggetti come ad eventi che riportano un certo avvenimento nell'applicazione. Di solito vengono chiamati dopo un input, ovvero quando si verifica un evento specifico nell'applicazione, come un click del mouse o quando un pulsante viene premuto.
	\item \textbf{Store (archivio)}: l'oggetto centrale che contiene lo stato dell'applicazione in un determinato momento.
	\item \textbf{Reducers}: funzioni che descrivono esattamente come deve essere cambiato lo stato, in risposta alle azioni chiamate sullo store. Come parametri di ingresso accettano lo stato corrente e l'azione che si vuole eseguire, ritornando un nuovo stato. I reducer possono essere paragonati a dei \textit{listener} che gestiscono gli eventi (\textit{actions}) in base al loro tipo.
	\item \textbf{Dispatch}: metodo di cui lo store dispone. L'unico modo per aggiornare lo stato è invocando questa funzione, passando come parametro un'azione.  A questo punto lo store può eseguire il suo \textit{reducer} e salvare il nuovo stato al suo interno. Nel momento in cui un evento viene innescato (proveniente per esempio della \ac{UI}), si vuole di conseguenza aggiornare il valore dello store.
	\item \textbf{Subscribe}: sono funzioni che permettono ai componenti grafici di ``iscriversi'' ai cambiamenti di stato dello store. Ogni volte che lo stato cambia, i componenti iscritti vengono avvisati, dando la possibilità di aggiornare la \ac{UI} in base al nuovo cambio di stato.
\end{itemize}
Insieme agli elementi appena descritti, la figura \cref{fig:redux-scheme} offre una visione più completa del funzionamento della gestione dello stato.
\begin{figure}[htb]
	\centering
	\includegraphics[scale=0.8]{imgs/Redux_Scheme.pdf}
	\caption{Funzionamento di uno store Redux}
	\label{fig:redux-scheme}
\end{figure}
Dallo schema è possibile notare come sia presente una unica direzione che collega le varie entità. Questo è dovuto al fatto che la sequenza dei passi da compiere per l'aggiornamento dello stato segue il paradigma \textbf{``one-way data flow"}. In particolare, per Redux si seguono quindi questi passaggi:
\begin{enumerate}	
	\item Qualcosa accade nell'applicazione, come un utente che fa clic su un pulsante.
	\item Viene chiamata la funzione \texttt{dispatch()}, specificando l'azione per modificare lo store Redux.
	\item Lo store esegue la funzione reducer con lo stato precedente e l'azione corrente, e salva il valore restituito come nuovo stato.
	\item Lo store notifica a tutte le parti dell'interfaccia utente che sono iscritte che lo store è stato aggiornato.
	\item Ciascun componente dell'interfaccia utente che necessita dei dati dallo store controlla se le parti dello stato di cui hanno bisogno sono cambiate.
	\item Ciascun componente che rileva che i suoi dati sono cambiati, forza un nuovo rendering con i nuovi dati, così da poter aggiornare ciò che viene mostrato sullo schermo.
\end{enumerate}
\paragraph{Store di un nodo}
All'interno dell'interfaccia grafica è possibile ottenere le informazioni di un nodo effettuando una apposita query al server GraphQL. Queste informazioni possono essere utili all'interno di tutta l'interfaccia grafica e sono appositamente memorizzate in uno store. \textit{KVision} offre un'implementazione ad-hoc per gli store Redux. La creazione avviene come riportato in \cref{lst:redux-node-store}.
\lstinputlisting[float=htb, language=Kotlin, label={lst:redux-node-store}, caption={Creazione dello store Redux \texttt{nodeStore} nel framework KVision}]{listings/Store.kt}
\texttt{createTypedReduxStore} è una funzione della libreria di \textit{KVision} che permette la creazione di uno store alla quale, da definizione, vengono passati il relativo reducer e lo stato iniziale.

Lo stato dell'applicazione è memorizzato all'interno dello store Redux. Non può essere cambiato direttamente dall'esterno, questo perché vige una regola di immutabilità. Lo stato corrente è sempre un oggetto immutabile, del quale non si possono cambiare i contenuti. Per aggiornare i valori in modo immutabile, devono essere eseguite copie degli oggetti esistenti e solo dopo modificare le copie. Dal \cref{lst:redux-node-state} si può notare come la classe è di tipo \texttt{data}. Dopotutto lo stato deve ``solo'' contenere informazioni.

\lstinputlisting[float=htb, language=Kotlin, label={lst:redux-node-state}, caption={Classe \texttt{NodeState} che modella lo state}]{listings/StoreState.kt}

Per questo store pertanto, le azioni sono definite come classi che possono contenere informazioni aggiuntive. La classe dichiarata è di tipo \texttt{sealed}, il che permette di usare l'espressione \texttt{when} in modo esaustivo in fase di definizione delle operazioni per tipo. Al suo interno, vi sono le sottoclassi che contengono le azioni vere e proprie. Il \cref{lst:redux-node-action} mostra come \texttt{NodeAction} è stata implementata.

\lstinputlisting[float=htb, language=Kotlin, label={lst:redux-node-action}, caption={Classe action per lo store \texttt{NodeStore}}]{listings/StoreAction.kt}

Infine, il \cref{lst:redux-node-reducer} mostra come è stata definita la funzione reducer, che è l'unico modo per modificate lo stato. Di solito, viene chiamata dopo che un'azione è stata inviata allo store.

\lstinputlisting[float=htb, language=Kotlin, label={lst:redux-node-reducer}, caption={Funzione reducer per lo store \texttt{NodeStore}}]{listings/StoreReducer.kt}
Facendo click sul canvas sopra a un nodo, viene quindi chiamata la funzione \texttt{dispatch()} per aggiornare lo stato corrente.

\lstinputlisting[float=htb, language=Kotlin, label={lst:redux-node-query}, caption={Chiamata alla query \texttt{nodeQuery} con l'\texttt{id} recuperato dall'evento click}]{listings/NodeById.kt}


\paragraph{Aggiornamento della componente grafica}
Il framework \textit{KVision} viene di nuovo in aiuto per l'aggiornamento di una componente grafica in base al cambiamento di stato di uno store. Viene utile a questo scopo la concezione di \textit{state binding}. Si riferisce al processo di collegamento che avviene tra uno stato dell'applicazione e lo stato di una componente della \ac{UI}.  Questo permette al componente di reagire dinamicamente ai cambiamenti nello stato, aggiornando automaticamente la visualizzazione quando lo stato cambia. In \textit{KVision}, questo meccanismo è accessibile in diversi modi tramite funzioni specifiche fornite dalla libreria, che consentono di associare direttamente il valore dello stato a un componente \ac{UI} senza dover gestire manualmente gli aggiornamenti della visualizzazione. Il \cref{lst:redux-state-binding} mostra lo state binding tra una parte (semplificata) della componente grafica \texttt{NodeProperties} (\cref{item:node-inspection}) e lo store \texttt{NodeStore} descritto in \cref{lst:redux-node-store}. I due contenitori HTML \texttt{div} riportano le proprietà del nodo selezionato, rispettivamente il codice identificativo \texttt{id} e la coppia di coordinate (x, y) nel canvas.
La funzione \texttt{bind()} esegue il processo descritto fino ad'ora. In alternativa, sarebbe stato comunque possibile aggiornare la componente grafica invocando la funzione \texttt{subscribe()} offerta da \texttt{NodeStore}. 

\lstinputlisting[float=htb, language=KVision, label={lst:redux-state-binding}, caption={Binding tra i componenti \texttt{div} e \texttt{NodeStore}}]{listings/StateBinding.kt}

\section{Rappresentazione dei nodi}
Nella \cref{section:rendering-analysis} abbiamo analizzato come la scelta di un canvas sia stato l'approccio più adatto per motivi di performance. Per questo motivo è stato impiegato l'oggetto \textit{canvas} di HTML, o meglio, si è usata la sua implementazione da parte di \textit{KVision}. All'interno del \textit{canvas} è possibile accedere al suo contesto tramite la proprietà \texttt{context2D}, di tipo \texttt{CanvasRenderingContext2D}, che fornisce la capacità di disegnare qualsiasi cosa in un piano bidimensionale, combinando linee, archi, curve e altro. \texttt{CanvasRenderingContext2D} sfrutta le capacità della libreria \textit{WebGL} \footnote{\url{https://developer.mozilla.org/en-US/docs/Web/API/WebGL_API}}, un \ac{API} JavaScript per il rendering di grafiche 2D (e 3D) all'interno di un qualsiasi browser. Questo standard aderisce strettamente a quello OpenGL ES 2.0, dando la possibilità a queste \ac{API} di trattare completo vantaggio dall'utilizzo dell'accelerazione grafica del sistema.
Ora è possibile fornire lo schema in \cref{fig:sequence-rendering-advance}, una versione più dettagliata di quello in \cref{section:rendering-analysis}. 

\begin{figure}[htb]
	\centering
	\includegraphics[scale=0.8]{imgs/Sequenza_rendering_plus.pdf}
	\caption{Diagramma di sequenza per il disegno dei nodi nel canvas}
	\label{fig:sequence-rendering-advance}
\end{figure}
Seguendo lo schema, viene effettuata una operazione di tipo \textbf{subscription} al server GraphQL per il recupero delle posizioni dei nodi. 
\lstinputlisting[float=htb, language=Kotlin, label={lst:node-subscription}, caption={Chiamata alla subscription sulla posizione dei nodi}]{listings/SubscriptionNodes.kt}
Il \cref{lst:node-subscription} mostra la chiamata alla funzione che esegue l'operazione di \textbf{subscription}. La funzione restituisce il risultato della query come un oggetto di tipo \texttt{Flow}. Quest'ultima è una classe di Kotlin che rappresenta una sequenza di valori che vengono prodotti in modo asincrono. Più nello specifico, viene restituito un \textit{cold} \texttt{Flow}. Questo significa che i dati non vengono forniti  fino a quando il flusso non viene consumato. I valori possono essere consumati in un secondo momento, come avviene nel  \cref{lst:node-subscription-collection}. Notare come in questo caso il \textit{triggering event} per l'aggiornamento dello store legato alla posizione dei nodi è proprio la funzione \texttt{collect()}.
\lstinputlisting[float=htb, language=Kotlin, label={lst:node-subscription-collection}, caption={Consumazione della subscription e aggiornamento dello store dei nodi}]{listings/SubscriptionNodesCollection.kt}
 
 A questo punto, è necessario iscriversi ai cambiamenti di stato e richiamare la funzione \textit{redraw} del canvas, come nel \cref{lst:node-subscription-subscribe-redraw}. Il meccanismo di state binding in questo caso è reso possibile attraverso l'utlizzo della funzione \texttt{subscribe()} dello store \texttt{EnvironmentStore.store}.
\lstinputlisting[float=htb, language=Kotlin, label={lst:node-subscription-subscribe-redraw}, caption={Iscrizione agli aggiornamenti dello store e richiamo della funzione \texttt{redrawNodes()}}]{listings/NodeSubscribeRedraw.kt}

%Sebbene questo elemento necessiti dell'impiego di più codice bla bla bla...
\section{Verifica}